\documentclass[11pt]{report}         
\usepackage{pxfonts}
\usepackage{makeidx}
\usepackage[plainpages=false,pdfpagelabels]{hyperref}
\usepackage{graphicx}
\usepackage{longtable}
\usepackage{pdflscape}
\usepackage[round,sort]{natbib}

\bibliographystyle{abbrvnat}

\title{Technical Manual for University of Texas Site Response (UTSR)}
\author{Albert R. Kottke and Ellen M. Rathje}
\date{\today}

% Create the index
\makeindex

\begin{document}
\maketitle
\tableofcontents
\listoffigures
\listoftables

\chapter{Introduction}
The computer program University of Texas Site Response (UTSR) was written in 2007 by Albert Kottke
and Professor Ellen M. Rathje.  (mention financial support?) UTSR performs equivalent linear site
response analysis in the frequency domain using time domain input motions or random vibration theory
(RVT) methods, and allows for randomization of the site properties. The following document provides
instruction on using UTSR, as well as the computational details of the analysis.

UTSR is distributed under the GNU General Public License which can be found here:
\texttt{http://www.gnu.org/licenses/}.

\chapter{Equivalent Linear Site Response Analysis}
UTSR uses an iterative linear wave propagation with strain dependent dynamic properties.  This
method is referred as equivalent linear and was first used in the computer program SHAKE (ref?).
UTSR only computes the response for vertically propagating, horizontally polarized shear waves
propagated through a site with horizontal layers.

The following chapter introduces strain dependent soil properties, linear wave propagation through a
layered medium, and the equivalent linear site response analysis method.

\section{Linear Wave Propagation}\label{ch:sra:waveProp}
For linear elastic, one-dimensional wave propagation, the soil is assumed to behave as a
Kelvin-Voigt solid, in which the dynamic response is described using a purely elastic spring and a
purely viscous dashpot \citep{kramer:96}.  The solution to the one-dimensional wave
equation\index{wave equation} for a single wave frequency ($\omega$) provides displacement ($u$) as
a function of depth ($z$) and time ($t$):
\begin{equation}
  u(z,t) = A\exp\left[i\left( \omega t + k^* z \right)\right] + B\exp\left[i\left( \omega t - k^* z
  \right)\right] 
  \label{eq:waveEq}
\end{equation}
In equation (\ref{eq:waveEq}), $A$ and $B$ represent the respective amplitudes of the upward ($-z$)
and downward ($+z$) waves, respectively (Figure~\ref{fig:sraSimpleWave}).  The complex wave number
($k^*$) in equation (\ref{eq:complexWaveNo}) is related to the shear modulus ($G$), damping ratio
($\xi$), and mass density ($\rho$) of the soil using:
\begin{equation}
 	k^* 	= \frac{\omega}{v_s^*} \\
	\label{eq:complexWaveNo}
\end{equation}
\begin{equation}
	v_s^*	= \sqrt{ \frac{G^*}{\rho} } \\
\end{equation}
\begin{equation}
	G^* 	= G \left( 1 - 2\xi^2 + \imath 2 \xi \sqrt{ 1 - \xi^2 }\right) 
		\simeq G ( 1 + \imath 2 \xi ) \\
	\label{eq:complexShear}
\end{equation}
$G^*$ and $v_s^*$ are called the complex shear modulus\index{shear modulus!complex}and complex
shear-wave velocity, respectively.  If the damping ratio ($\xi$) is small ($<10-20\%$), then the
approximation of the complex shear modulus in equation (\ref{eq:complexShear}) is appropriate.  UTSR
uses the complete definition of the complex shear-modulus, not the approximation, in the
calculations.

\begin{figure}[tb]
  \begin{center}
	\includegraphics[width=0.7\linewidth]{figures/sraSimpleWave.pdf}
  \end{center}
  \caption{The notation used in the wave equation}
  \label{fig:sraSimpleWave}
\end{figure}

Equation (\ref{eq:waveEq}) applies only to a single layer with uniform soil properties and the
wave amplitudes ($A$ and $B$) can be computed from the layer boundary conditions.  For a layered
system, shown in Figure~\ref{fig:sraNomenclature}, the wave amplitudes are calculated using
recursive formulas developed by maintaining: compatibility of displacement and shear stress at the
layer boundaries.  Using these assumptions, the following recursive formulas are developed:
\begin{equation}
  \begin{array}{rcl}
	A_{m+1} & = & \frac{1}{2} A_m \left( 1 + \alpha_m^* \right) \exp\left(\imath k_m^* h_m \right) +
		\frac{1}{2} B_m \left( 1 - \alpha_m^* \right) \exp\left(-\imath k_m^* h_m \right)\\
	B_{m+1} & = & \frac{1}{2} A_m \left( 1 - \alpha_m^* \right) \exp\left(\imath k_m^* h_m\right) +
		\frac{1}{2} B_m \left( 1 + \alpha_m^* \right) \exp\left(-\imath k_m^* h_m \right)\\
  \end{array}
\end{equation}
where $h_m$ is the layer height and $\alpha_m^*$ is the complex impedance ratio.  The complex
impedance ratio is defined as:
\begin{equation}
  \alpha_m^* = \frac{k_m^* G_m^*}{k_{m+1}^* G_{m+1}^*} = \frac{\rho_m ( v_s^* )_m }{\rho_{m+1} (
  v_s^* )_{m+1} } 
\end{equation}
and quantifies the relative amplitudes of the upward and downward waves.  At the surface of the soil
column ($m=1$), the shear stress must equal zero, therefore the amplitudes of the upward and
downward waves must be equal ($A_1=B_1$).  

\begin{figure}
  \begin{center}
    \setlength{\unitlength}{0.6cm}
	\begin{picture}(12,9)
	  % Layers
	  \put(0,1){\line(1,0){12}}
	  \put(0,3){\line(1,0){12}}
	  \put(0,4){\line(1,0){12}}
	  \put(0,5){\line(1,0){12}}
	  \put(0,7){\line(1,0){12}}
	  \put(0,8){\line(1,0){12}}
	  \put(0,9){\line(1,0){12}}
	  % Wave Vectors
	  \thicklines
	  \put(3.9,1){\vector(0,-1){0.75}}
	  \put(3.9,4){\vector(0,-1){0.75}}
	  \put(3.9,5){\vector(0,-1){0.75}}
	  \put(3.9,8){\vector(0,-1){0.75}}
	  \put(3.9,9){\vector(0,-1){0.75}}
	  \put(3.4,0.25){\vector(0,1){0.75}}
	  \put(3.4,3.25){\vector(0,1){0.75}}
	  \put(3.4,4.25){\vector(0,1){0.75}}
	  \put(3.4,7.25){\vector(0,1){0.75}}
	  \put(3.4,8.25){\vector(0,1){0.75}}
	  \put(2.2,8.3){$A_1$}
	  \put(2.2,7.3){$A_2$}
	  \put(2.2,4.3){$A_m$}
	  \put(1.9,3.3){$A_{m+1}$}
	  \put(2.2,0.3){$A_n$}
	  \put(4.1,8.3){$B_1$}
	  \put(4.1,7.3){$B_2$}
	  \put(4.1,4.3){$B_m$}
	  \put(4.1,3.3){$B_{m+1}$}
	  \put(4.1,0.3){$B_n$}

	  % Layer indicators
	  \put(0.0,8.3){$1$}
	  \put(0.0,7.3){$2$}
	  \put(0.0,4.3){$m$}
	  \put(0.0,3.3){$m+1$}
	  \put(0.0,0.3){$n$}
	  % Layer properties
	  \put(6,8.3){\(\rho_{1} \  h_{1} \  G_{1} \  \xi_{1} \)}
	  \put(6,7.3){\(\rho_{2} \  h_{2} \  G_{2} \  \xi_{2} \)}
	  \put(6,4.3){\(\rho_{m} \  h_{m} \  G_{m} \  \xi_{m} \)}
	  \put(6,3.3){\(\rho_{{m+1}} \  h_{{m+1}} \  G_{{m+1}} \  \xi_{{m+1}} \)}
	  \put(6,0.3){\(\rho_{n} \  h_{n} \  G_{n} \  \xi_{n} \)}
	\end{picture}
  \end{center}
  \caption{Nomenclature for the theoretical wave propagation.}
  \label{fig:sraNomenclature}
\end{figure}

The wave amplitudes ($A$ and $B$) within the soil profile are calculated at each frequency (assuming
known stiffness and damping within each layer) and used to computed the response at the surface of a
site.  This calculation is performed by setting $A_{1}=B_{1}=1.0$ at the surface and iteratively
calculating the wave amplitudes ($A_{m+1}$,$B_{m+1}$) in successive layers until the input (base)
layer is reached.  The transfer function between the motion in the layer of interest ($m$) and in
the rock layer ($n$) at the base of the deposit is defined as:
\begin{equation}
  TF_{m,n}(\omega) = \frac{u_m(\omega)}{u_n(\omega)}= \frac{A_m + B_m}{A_n + B_n}
  \label{eq:transFunc}
\end{equation}
where $\omega$ is the frequency of the harmonic wave.  The transfer function is the ratio of the
amplitude of motion--either displacement, velocity, or acceleration--between two layers of interest
and varies with frequency.  The transfer function for the site with the properties presented in
Table~\ref{tab:sraSite} is shown in Figure~\ref{fig:sraAccelTfOutcropWithin}.  The locations of the
peaks in the transfer function are controlled by the modes of vibration of the soil deposit.  The
peak at the lowest frequency represents the fundamental (i.e. first) mode of vibration and results
in the largest amplification.  The peaks at higher frequencies are the higher vibrational modes of
the site.  The first natural frequency of a site is inversely related to the site period, where the
site period is defined as \citep{kramer:96}:
\begin{equation}
  T_s = \frac{4\cdot h_{soil}}{\overline{v}_s}\\ 
  \label{eq:sitePeriod}
\end{equation}
In equation (\ref{eq:sitePeriod}), $h_{soil}$ is the total height of the soil and $\overline{v}_s$
average velocity of the site.  For the example site (Table~\ref{tab:sraSite}), the site period is
calculated to be 0.57 s which corresponds to a natural frequency of 1.75 Hz.  In the transfer
function (Figure~\ref{fig:sraAccelTfOutcropWithin}), the peak with the highest amplification occurs
at this frequency.  The amplitudes of the peaks are controlled by the damping within the soil.  As
the damping of the system increases, the amplitudes of the peaks decrease, which results in less
amplification at the surface.

\begin{table}[t]
  \centering
  \begin{tabular}{llc}
	\hline\hline
	\textbf{Property} & \textbf{Rock} & \textbf{Soil} \\
	\hline
	Mass Density ($\rho$) 		& 2.24 g/cm$^3$ & 1.93 g/cm$^3$ \\
	Height ($h$)				& $\inf$		& 50 m \\
	Shear-wave Velocity ($v_s$)	& 1500 m/s		& 350 m/s \\
	Damping ratio ($\xi$)		& 1\%			& 7\% \\
	\hline
  \end{tabular}
  \caption{The site properties of an example site.}
  \label{tab:sraSite}
\end{table}
\begin{figure}[t]
  \begin{center}
	\includegraphics[width=\linewidth]{figures/sraAccelTfOutcropWithin.pdf}
  \end{center}
  \caption{The input (\emph{within}) to surface (\emph{outcrop}) transfer function site in
  Table~\ref{tab:sraSite}.}
  \label{fig:sraAccelTfOutcropWithin}
\end{figure}


The response at the layer of interested is computed by multiplying the Fourier spectrum of the
input rock motion by the transfer function\index{transfer function!acceleration}:
\begin{equation}
  Y_m(\omega) = TF_{m,n}(\omega) \cdot Y_n(\omega)
  \label{eq:tfApplication}
\end{equation}
where $Y_n$ is the input Fourier amplitude spectrum at layer $n$ and $Y_m$ is the Fourier amplitude
spectrum at the top of the layer of interest.  The Fourier Spectrum of the input motion can defined
a variety of methods and is discussed further in Section FIXME.

One issue that must be considered is that the input Fourier spectrum is typically the response at
the ground surface, where the upgoing and downgoing wave amplitudes are equal, not at the base of a
soil deposit, where the wave amplitudes are not equal (Figure~\ref{fig:sraOutcropWithin}). However,
the motion is applied as input at the base of a soil deposit and the change in boundary conditions (
$A_n = B_n$ for the surface, $A_n \ne B_n$ at the base of a soil deposit) must be taken into
account.  The motions at any free surface are referred to as \emph{outcrop}\index{motion
type!outcrop} records and their amplitudes are described by twice the amplitude of the upward wave
($2A$).  To apply equation (\ref{eq:transFunc}) to an \emph{outcrop} motion a second transfer
function is required to translate from an \emph{outcrop} motion to a \emph{within} motion.  The
combined transfer function is defined as:
\begin{equation}
  TF_{m,n}(\omega) = \underbrace{\frac{A_n + B_n}{2\cdot A_n}}_{outcrop \to within} \cdot
  \underbrace{\frac{A_m + B_m}{A_n + B_n}}_{within \to layer_n} = \underbrace{\frac{A_m + B_m}{ 2
  \cdot A_n }}_{outcrop \to layer_n}
\end{equation}
Motions recorded at depth (e.g. recorded in a borehole) are referred to as
\emph{within}\index{motion type!within} motions and for these motions the transfer function given in
equation (\ref{eq:transFunc}) can be used.  Figure~\ref{fig:sraAccelTfOutcropOutcrop}, shows the
transfer function for the site profile presented in Table~\ref{tab:sraSite} and applying the motions
as if it were an outcrop recording.  In comparison with \emph{within}-to-\emph{outcrop} transfer
(Figure~\ref{fig:sraAccelTfOutcropWithin}), the amplification for all modes is decreased
substantially. In the same manner, the response at the layer of interest can be computed as an
\emph{outcrop} or a \emph{within} response by using the appropriate wave amplitudes.  

\begin{figure}[tb]
  \begin{center}
	\includegraphics[width=0.7\linewidth]{figures/sraWithinOutcrop.pdf}
  \end{center}
  \caption{\emph{Outcrop} describes upward and downward waves being equal, \emph{within} is used
  when the upward and downward waves are not equal.}
  \label{fig:sraOutcropWithin}
\end{figure}
\begin{figure}[tb]
  \begin{center}
	\includegraphics[width=\linewidth]{figures/sraAccelTfOutcropOutcrop.pdf}
  \end{center}
  \caption{The input (\emph{outcrop}) to surface (\emph{outcrop}) transfer function site in
  Table~\ref{tab:sraSite}.}
  \label{fig:sraAccelTfOutcropOutcrop}
\end{figure}
\clearpage

\section{Equivalent-Linear Analysis}\label{ch:sra:equivLinear}
The previous section assumed that the soil was linear-elastic. However, soil is nonlinear, such that
the properties of soil (shear modulus, $G$, and damping ratio, $\xi$) vary with shear strain, and
thus the intensity of shaking.  In equivalent-linear site response analysis, the nonlinear response
of the soil is approximated by modifying the linear elastic properties of the soil based on the
induced strain level.  Because the induced strains depend on the soil properties, the strain
compatible shear modulus and damping ratio values are iteratively calculated based on the computed
strain.  

A transfer function is used to compute the shear strain in the layer.  In the calculation of the
transfer function, the shear strain is computed at the middle of the layer ($z=h_m/2$) and used to
select the strain compatible soil properties.  Unlike the previous transfer functions that merely
amplified the Fourier amplitude spectrum, the strain transfer function\index{transfer
function!strain} amplifies the motion and converts acceleration into strain.  The strain transfer
function based on an outcropping input motion is defined by:
\begin{equation}
  TF_{mn}^{strain}(\omega) = \frac{\gamma(\omega, z=h_m/2)}{\ddot{u}_{n,outcrop}(\omega)} =
  \frac{\imath k_m \left[
  A_m \exp\left( \imath k_m^* h_m / 2 \right) - B_m \exp\left( -\imath k_m^* h_m / 2 \right) \right]
  }{-\omega^2 \left( 2 \cdot A_n \right)}
  \label{eq:strain_tf}
\end{equation}
The strain Fourier amplitude spectrum within a layer is calculated by applying the strain transfer
function to the Fourier amplitude spectrum of the input motion.  The maximum strain within the layer
is computed from this Fourier amplitude spectrum (further discussed in Chapter~\ref{ch:srm}).  However, it
is not appropriate to use the maximum strain within the layer to compute the strain-compatible soil
properties, because the maximum strain only occurs for an instant in time.  Instead, an effective
strain ($\gamma_{\mathrm{eff}}$) is calculated from the maximum strain.  Typically, the effective
strain is 65\% of the maximum strain. 

Equivalent-linear site response analysis requires that the strain dependent nonlinear properties
(i.e. $G$ and $\xi$) be defined.  The initial (small strain) shear modulus\index{shear
modulus!maximum (small strain)} ($G_{max}$) is calculated by:
\begin{equation}
  G_{max} = \rho v_s ^ 2
\end{equation}
where $\rho$ is the mass density of the soil, and $v_s$ is the measured shear-wave velocity.
Characterizing the nonlinear behavior of $G$ and $\xi$ is achieved through modulus reduction and
damping curves that describe the variation of $G/G_{max}$ and $\xi$ with shear strain (discussed in
the next section).  Using the initial dynamic properties of the soil, equivalent-linear site
response analysis involves the following steps:
\begin{enumerate}
  \item The wave amplitudes ($A$ and $B$) are computed for each of the layers
  \item The strain transfer function is calculated for each of the layers.
  \item The maximum strain within each layer is computed by applying the strain transfer function
	to the input Fourier amplitude spectrum and finding the maximum response (see
	Chapter~\ref{ch:srm}).
  \item The effective strain ($\gamma_{\mathrm{eff}}$) is calculated from the maximum strain within
	each layer.
  \item The strain compatible shear modulus and damping ratio are recalculated based on the new
	  estimate of the effective strain within each layer.
  \item The new nonlinear properties ($G$ and $\xi$) are compared to the previous iteration and an
	error is calculated.  If the error for all layers is below a defined threshold the calculation
	stops.
\end{enumerate}
After the iterative portion of the program finishes, the dynamic response of the soil deposit is
computed.  

One method of looking at the response of the soil deposit is with a acceleration response
spectrum\index{response spectrum}, a tool used in structural analysis and to assess the frequency
content of the response.  An acceleration response spectrum is defined as the maximum responses of
single-degree-of-freedom oscillators with a range of natural periods and a specific damping ratio,
excited by a given acceleration-time history.  The acceleration response spectrum transfer function
is defined as:
\begin{equation}
	  H(f) = \frac{-f_n^2}{(f^2-f_n^2) - 2 \cdot \imath \cdot \xi \cdot f_n \cdot f}
\end{equation}
where $f_n$ is the natural frequency of the oscillator, $\xi$ is the damping ratio, and $f$ is the
frequency. (FIXME uses $f$ instead of $\omega$) The damping ratio is typically defined at 5\% as
this corresponds to the approximate damping of an undamaged building.  The response at
single period by applying the transfer function to the Fourier spectrum of interest and 
computing the maximum response.  The response spectrum is computed by varying the natural frequency
of the oscillator over a range of interest, typically 0.01 to 10 seconds.

\begin{figure}[tb]
  \begin{center}
	\includegraphics[width=\linewidth]{figures/sraAccelSa.pdf}
  \end{center}
  \caption{The input and surface acceleration response spectrum ($\xi$=5\%).}
  \label{fig:sraAccelSa}
\end{figure}

\section{Dynamic Soil Properties}\label{ch:sra:dynprops}
In a dynamic system,  the properties that govern the response are the mass, stiffness, and damping.
In soil under seismic shear loading, the mass of the system is characterized by the mass density
($\rho$) and the layer height ($h$), the stiffness is characterized by the shear modulus ($G$), and
the damping is characterized by the viscous damping ratio ($\xi$).  The dynamic behavior of soil is
challenging to model because it is nonlinear, such that, both the stiffness and damping of the
system change with shear strain.  Section~\ref{ch:sra:equivLinear} introduced equivalent-linear
site response analysis in which the nonlinear response of the soil was simplified into a linear
system that used strain-compatible dynamic properties ($G$ and $\xi$).  The analysis requires that
the strain dependence of the nonlinear properties within a layer be fully characterized.

Defining the mass density of the system is a straight forward process because the density falls
within a limited range for soil and a good estimate of the mass density can be made based on soil
type alone.  Characterization of the stiffness and damping properties of soil is more complicated,
the most rigorous approach requiring testing in both the field and laboratory.  

The shear modulus and material damping of the soil are characterized using the small strain shear
modulus ($G_{max}$), modulus reduction curves that relate $G/G_{max}$ to shear strain, and damping
ratio curves that relate $\xi$ to shear strain.  The small strain shear modulus is best
characterized by in situ measurement of the shear-wave velocity as a function of depth.  An example
shear-wave velocity profile is shown in Figure~\ref{fig:sraVsProfile}. The profile tends to be
separated into discrete layers associated with different geologic strata with a generally increasing
shear-wave velocity with increasing depth.  Examples of modulus reduction and damping curves for
soil are shown in Figure~\ref{fig:sraNLLab}.  These curves show a decrease in the soil stiffness and
an increase in the damping ratio with an increase in shear strain. 

\begin{figure}[tb]
	\begin{center}
		\includegraphics{figures/sraVsProfile.pdf}
	\end{center}
	\caption{An example shear-wave velocity profile}
	\label{fig:sraVsProfile}
\end{figure}
\begin{figure}[tb]
	\centering
	\includegraphics[width=\textwidth]{figures/sraNLLab.pdf}
	\caption{Examples of shear modulus reduction and material damping curves for soil.}
	\label{fig:sraNLLab}
\end{figure}

The modulus reduction and damping curves may be obtained from laboratory measurements on soil
samples or derived from empirical models based on soil type and other variables.  One of the most
comprehensive empirical models was developed by \citet{darendeli:01}.  The model expands on the
hyperbolic model presented by \citet{hardin:72} and accounts for the effects of confining pressure
($\sigma_0'$), plasticity index ($PI$), over-consolidation ratio ($OCR$), frequency ($f$), and
number of cycles of loading ($N$) on the modulus reduction and damping curves. 

In the \citet{darendeli:01} model, the shear modulus reduction curve is a hyperbola defined by:
\begin{equation}
  \frac{G}{G_{max}} = \frac{1}{1 + \left( \frac{\gamma}{\gamma_r} \right)^{a}}
  \label{eq:shearmod}
\end{equation}
where $a$ is 0.9190, $\gamma$ is the shear strain and $\gamma_r$ is the reference shear strain.  The reference
shear strain is computed from:
\begin{equation}
  \gamma_r = \left(\frac{\sigma_0'}{p_a}\right)^{0.3483} \left( 0.0352 + 0.0010 \cdot PI \cdot OCR^{0.3246} \right)
\end{equation}
where $\sigma_0'$ is the mean effective stress and $p_a$ is the atmospheric pressure in atm. In the
model, the damping ratio is calculated from the minimum damping ratio at small strains ($\xi_{min}$)
and from the damping associated with hysteretic Masing behavior ($\xi_{Masing}$).  The minimum damping is
calculated from:
\begin{equation}
  \xi_{min}(\%) = (\sigma_0')^{-0.2889} \left( 0.8005 + 0.0129 \cdot PI \cdot OCR ^{-0.1069} \right) \left[
  1 + 0.2919 \ln\left( f \right) \right]
  \label{eq:dmin}
\end{equation}
where $f$ is the excitation frequency (Hz). The computation of the Masing damping requires the
calculation of the area within the stress-strain curve predicted by the shear modulus reduction
curve.  The integration can be approximated by:
\begin{equation}
  \xi_{Masing}(\%) = c_1 \xi_{Masing,a=1} + c_2 \xi_{Masing,a=1} + c_3 \xi_{Masing,a=1}
  \label{eq:dmasing}
\end{equation}
where:
\begin{equation}
  	\xi_{masing,a=1}(\%) = \frac{100}{\pi} \left\{ 4 \left[ \frac{\gamma - \gamma_r\ln\left( \frac{\gamma +
	\gamma_r}{\gamma_r} \right)}{\frac{\gamma^2}{\gamma+\gamma_r}} \right] - 2 \right\} \\
\end{equation}
\begin{equation}
  \begin{array}{rcl}
  	c_1 & = & -1.1143 a^2 + 1.8618 a + 0.2533 \\
	c_2 & = & 0.0805 a^2 - 0.0710 a - 0.0095 \\
	c_3 & = & -0.0005 a^2 + 0.0002 a + 0.0003 \\
  \end{array}
\end{equation}
The minimum damping ratio in equation (\ref{eq:dmin}) and the Masing damping in equation
(\ref{eq:dmasing}) are combined to compute the total damping ($\xi$) using:
\begin{equation}
  \xi = b  \left( \frac{G}{G_{max}} \right)^{0.1} \cdot \xi_{Masing} + \xi_{min} 
  \label{eq:damping}
\end{equation}
where $b$ is defined as:
\begin{equation}
  b = 0.6329 - 0.0057 \ln\left( N \right)
\end{equation}
where $N$ is the number of cycles of loading.  In most site response applications, the number of
cycles ($N$) and the excitation frequency ($f$) in the model are typically defined as 10 and 1,
respectively.  Figure~\ref{fig:sraNLEmpirical} shows the predicted nonlinear curves for a sand
($PI=0, OCR=1$) at an effective confining pressure of 1 atm.

\begin{figure}[tb]
  \begin{center}
	\includegraphics[width=\linewidth]{figures/sraNLEmpirical.pdf}
  \end{center}
  \caption{The nonlinear soil properties predicted by the \citet{darendeli:01} model.}
  \label{fig:sraNLEmpirical}
\end{figure}

A Bayesian approach was used in the \citet{darendeli:01} model to calculate the model coefficients.
One of the unique aspects of this model is that the scatter of the data about the mean estimate is
quantified.  In the \citet{darendeli:01} model, the uncertainty about the mean value is assumed to
be normally distributed.  The normal distribution is described using a mean and standard deviation.
The mean values are calculated from equations (\ref{eq:shearmod}) and (\ref{eq:damping}). The
standard deviation is a function of the amplitude of the nonlinear property (i.e. $G/G_{max}$ and
$\xi$).  The standard deviation of the normalized shear modulus ($\sigma_{NG}$) is computed by:
\begin{equation}
  \sigma_{NG} = \exp(-4.23) + \sqrt{ \frac{0.25}{\exp(3.62)} - \frac{\left(G/G_{max} -
  0.5\right)^2}{\exp(3.62)} }
  \label{eq:sigmaShear}
\end{equation}
This model results in small $\sigma_{NG}$ when $G/G_{max}$ is close to 1 or 0 and relatively large
$\sigma_{NG}$ when $G/G_{max}$ is equal to 0.5. The standard deviation of the damping
($\sigma_{\xi}$) is computed by:
\begin{equation}
  \sigma_{\xi} = \exp(-5.0) + \exp(-0.25) \sqrt{\xi (\%)}
  \label{eq:sigmaDamping}
\end{equation}
In the damping uncertainty model, the $\sigma_{\xi}$ increases with increasing damping.  Using these
definitions for the standard deviation, the $\pm\sigma$ modulus reduction and damping curve for sand
at a confining pressure of 1 atm are shown in Figure~\ref{fig:sraNLEmpiricalSigma}.

\begin{figure}[tb]
  \begin{center}
	\includegraphics[width=\linewidth]{figures/sraNLEmpiricalSigma.pdf}
  \end{center}
  \caption{The mean and mean $\pm\sigma$ nonlinear soil properties predicted by \citet{darendeli:01}.}
  \label{fig:sraNLEmpiricalSigma}
\end{figure}
\clearpage 

\chapter{Site Response Methods}\label{ch:srm}
\section{Time Domain}
By applying the
inverse FFT to the strain Fourier amplitude spectrum, the strain-time history is computed
(Figure~\ref{fig:sraStrainTH}).  From this strain-time history, the maximum strain ($\gamma_{max}$)
is found.  


\begin{figure}[tb]
  \begin{center}
	\includegraphics[width=\linewidth]{figures/sraStrainTH.pdf}
  \end{center}
  \caption{An example of the strain-time history and effective strain ($\gamma_{\mathrm{eff}}$).}
  \label{fig:sraStrainTH}
\end{figure}

The acceleration response spectra for the input and surface acceleration-time histories in Figure
FIXME are shown in Figure~\ref{fig:sraAccelSa}.  Comparing the response spectra indicates that there
is significant amplification of the input motion at periods less than 1 second.  The largest
amplification is observed at 0.57 s, which represents the fundamental period of the site.


\section{Random Vibration Theory}


%% Force Bibliography to appear in contents
\newpage
\addcontentsline{toc}{chapter}{Bibliography} 
\bibliography{references}

%% Create an index and add it to the table of contents
\newpage
\addcontentsline{toc}{chapter}{Index} 
\printindex

\end{document}


